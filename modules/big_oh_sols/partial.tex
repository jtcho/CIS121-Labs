
% Lab 2 Handout

\labtitle{\semester}{Big-Oh}{TBD}

\section*{Lab Problem Solutions}

\subsection*{Problem 1}
Order the following functions such that if $f$ precedes $g$, then $f(n)$ is $O(g(n))$.

\begin{center}
    $\sqrt{n}$, $n$, $n^{1.5}$, $n^2$, $n \lg{n}$, $n\lg\lg{n}$, $n\lg{n^2}$, $2^{n/2}$, $2^n$, $\lg(n!)$, $n^2\lg{n}$, $n^3$, $2^{2^n}$
\end{center}

\subsubsection*{Solution.}
\begin{center}
    $\sqrt{n}$, $n$, $n\lg \lg n$, $n \lg n$, $\lg(n!)$, $n\lg{n^2}$, $n^{1.5}$, $n^2$, $n^2\lg{n}$, $n^3$, $2^{n/2}$, $2^n$, $2^{2^n}$
\end{center}

\subsection*{Problem 2}
Provide a runtime analysis of the following loop. That is, find both Big-Oh and Big$-\Omega$:

\begin{verbatim}
    for(int i = 0; i < n; i++)
        for (int j = i; j <= n; j++)
            for (int k = i; k <= j; k++)
                sum++;
\end{verbatim}

\subsubsection*{Solution.}

Observe that for fixed values of $i, j$, the innermost loop runs $\max\{1, j-i+1\} \leq n$ times. For instance, when $i = j = 0$, the innermost loop evaluates once. When $i = 0, j = n$, the innermost loop evaluates $n+1$ times. The middle loop runs a total number of $O((n-i)^2)$ times. Therefore, the entire block of code runs in $O(n^3)$.\\

To find a lower bound on the running time, we consider smaller subsets of values for $i,j$ and lower-bound the running time for the algorithm on these subsets. (A lower bound there would also be a lower bound for the original code's runtime!) Consider the values of $i$ such that $0 \leq i \leq n/4$ and values of $j$ such that $3n/4 \leq j \leq n$. For each of the $n^2/16$ possible combinations of these values of $i$ and $j$, the innermost loop runs at least $n/2$ times.
Therefore, the running time is at least $$(n^2/16)(n/2) = \Omega(n^3)$$

% The total number of different pairings of i and j values is the same as the total number of iterations of the outer two loops!

\subsubsection*{Alternate Solution for Big-Oh.}

One could solve this using exact sums, but we will leverage some Big-Oh notation. We know that the innermost loop runs in at most $(j-i+1)$ time for fixed $i,j$ (see other solution). Therefore the body of the middle loop runs at most $c(j-i+1)$ times. Therefore, we can express the runtime of the code shown as 

$$\sum_{i=0}^n \sum_{j=i}^n c(j-i+1) = O(n^3)$$.

\subsection*{Problem 3}
In this problem, you are \textbf{not} allowed to use the theorems about Big-Oh stated in the lecture notes. Your proof should follow exclusively from the definition of Big-Oh.\\

Prove or disprove the following statement:
\begin{center}
    $f(n) + g(n)$ is $\Theta(\max\big\{f(n),g(n)\big\}),$ where $f, g:R\rightarrow R^{+}$.
\end{center}

\subsubsection*{Solution.}

We show both Big-Oh and Big-$\Omega$ separately.\\

Let $c = 2, n_0 = 1$. $f(n) + g(n) \leq 2 \max\{f(n), g(n)\},\;\forall n \geq 1$. Therefore, $f(n) + g(n) = O(\max\{f(n), g(n)\})$.\\

Let $c = 1, n_0 = 0$. $f(n) + g(n) \geq \max\{f(n), g(n)\},\;\forall n \geq 0$. Therefore, $f(n) + g(n) = \Omega(\max\{f(n), g(n)\})$.\\

Therefore, as we have shown Big-Oh and Big-$\Omega$, we have shown Big-$\Theta$.

\subsection*{Problem 4}
Prove or disprove the following statement:
\begin{center}
    $2^n$ is $O(n!)$.
\end{center}

\subsubsection*{Solution.}

We want to show that $2^n \leq c(n!)\; \forall n \geq n_0$, for a positive real-valued $c$ and integer $n_0$.\\

\begin{proof}
    \begin{align*}
        2^n &\leq c(n!)\\
        \lg(2^n) &\leq \lg c + \lg(n!)\\
        n &\leq \lg c + \lg(n\cdot(n-1)\cdot(n-2)\dots2)\\
        n &\leq \lg c + \sum_{i=0}^n \lg i\\
        n \leq \sum_{i=n/2}^n \lg i &\leq \lg c + \sum_{i=0}^n \lg i\\
        n \leq n/2 \lg(n/2) &\leq \sum_{i=n/2}^n \lg i\\
        2 &\leq \lg(n/2)
    \end{align*}

    It is clear that choosing any $n_0 \geq 8$ and $c > 0$ will suffice! To check our work, let $n_0 = 8$ and $c = 1$.

    $$2^8 \leq 8! \rightarrow 256 < 40320$$

    A particularly obvious result, but it doesn't hurt to check.

\end{proof}

\subsection*{Problem 5}
Provide a runtime analysis of the following loop:

\begin{verbatim}
    for (int i = 2; i < n; i = i*i)
        for (int j = 1; j < Math.sqrt(i); j = j+j)
            System.out.println("*");
\end{verbatim}

\subsection*{Problem 6}
Prove or disprove the following statement:

\begin{center}
    $\lg(n!)$ is $\Theta(n\lg{n})$.
\end{center}

